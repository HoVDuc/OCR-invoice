\section*{\Large\centering Kết luận}{}
\addcontentsline{toc}{chapter}{Kết luận}
Sau quá trình nghiên cứu và tìm hiểu các phương pháp, thuật toán liên quan đến OCR cùng với sự nỗ lực của bản thân em đã hoàn thành đề tài kịp tiến độ đề tài ``\textbf{Nghiên cứu ứng dụng công nghệ OCR nhận dạng hóa đơn}'' và đạt được những kết quả sau:
\begin{enumerate}
    \item Tìm hiểu và áp dụng thành công các thuật toán ``state-of-the-art'' cho đề tài
    \item Huấn luyện thành công các mô hình cho nhiệm vụ khác nhau phát hiện văn bản, nhận dạng văn bản, \acrshort*{kie} và đạt hiệu quả tốt trên bộ dữ liệu thu thập được.
    \item Xây dựng hoàn chỉnh chương trình trích xuất thông tin hóa đơn.
    \item Thành công chuyển đổi ảnh hóa đơn thành dạng văn bản có thể xử lý được.
    \item Trích xuất được các thông tin quan trọng của hóa đơn như tên cửa hàng, địa chỉ, ngày mua, sản phẩm, \ldots Và lưu trữ thông tin dưới dạng văn bản có cấu trúc. 
\end{enumerate}

Như vậy, đề tài đã hoàn thành mục tiêu đề ra và đạt được những kết quả quan trọng trong việc ứng dụng công nghệ OCR để nhận dạng và trích xuất thông tin từ hóa đơn.

\subsection*{Những hạn chế của đề tài} \label{hanche}
Tuy vậy, đề tài còn một số hạn chế đáng chú ý như sau:
\begin{enumerate}
    \item Dễ ảnh hưởng bởi nhiều yếu tố như chất lượng hình ảnh, font chữ, định dạng
    hóa đơn
    \item Mô-đun phát hiện văn bản vẫn chưa phát hiện được hầu hết văn bản. Điều này đẫn đến thiếu xót thông tin trong quá trình nhận diện và trích xuất thông tin.
    \item Mô-đun nhận diện văn bản, dễ bị ảnh hưởng bởi chất lượng hình ảnh, và nhiễu. Điều này làm cho quá trình nhận dạng trở lên sai xót ảnh hưởng đến thông tin lưu trữ.
    \item Mẫu hóa đơn chưa được đa dạng và phong phú, chỉ nhận dạng trên vào mẫu hóa đơn nhất định.
\end{enumerate}

\subsection*{Hướng phát triển}
Trong tương lai, Em sẽ tiếp tục tìm hiểu và nghiên cứu thêm để khắc phục các vấn đề còn tồn tại của đề tài. Phát triển và mở rộng phạm vi nghiên cứu nhằm phát triển các kỹ năng và kiến thức để cải thiện bản thân. Dưới đây là những dự định phát triển:
\begin{enumerate}
    \item Khắc phục các hạn chế ở mục hạn chế, tối ưu hóa độ chính xác, thời gian chạy và đa dạng mẫu hóa đơn ,và cải thiện khả năng nhận dạng trong các trường hợp phức tạp.
    \item Xây dựng ứng dụng di động cho phép người dùng chụp ảnh hóa đơn bằng điện thoại di động và tự động đưa ra thông tin liên quan. Điều này sẽ giúp tạo ra một trải nghiệm tiện lợi và linh hoạt cho người dùng.
    \item Tích hợp ứng dụng OCR vào hệ thống quản lý tài liệu và dữ liệu của doanh nghiệp. Điều này sẽ giúp tự động hóa quy trình làm việc, giảm thiểu thời gian và nguy cơ sai sót trong việc nhập liệu thủ công.
\end{enumerate}

Trải qua quá trình nghiên cứu và thử nghiệm, Em nhận thấy rằng OCR có tiềm năng lớn trong việc tối ưu hóa quá trình nhận dạng và nhập liệu từ hóa đơn. Nhờ vào việc tự động nhận dạng ký tự, giảm thiểu sự phụ thuộc vào công việc thủ công, giảm nguy cơ sai sót từ việc nhập liệu thủ công và tăng khả năng đáp ứng nhanh chóng các yêu cầu kinh doanh.

Tổng kết lại, ``nghiên cứu về ứng dụng OCR trong nhận dạng hóa đơn'' là một bước tiến quan trọng trong việc áp dụng công nghệ vào thực tiễn kinh doanh. Tin rằng sự phát triển của OCR cùng với sự tương tác với các công nghệ khác sẽ mở ra những cơ hội mới và đóng góp tích cực cho sự phát triển của nền kinh tế và xã hội.




