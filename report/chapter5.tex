\chapter{Kết luận}
Trong báo cáo này, em đã tiến hành ``\textbf{Nghiên cứu về ứng dụng OCR trong việc nhận dạng hóa đơn}'' và đã thu được những kết quả khả quan và đáng chú ý. Bằng việc sử dụng công nghệ OCR, em đã thành công trong việc chuyển đổi các hình ảnh của hóa đơn giấy thành dữ liệu văn bản có thể xử lý được trên máy tính.

Trải qua quá trình nghiên cứu và thử nghiệm, Mô hình đã nhận thấy rằng OCR có tiềm năng lớn trong việc tối ưu hóa quá trình nhận dạng và nhập liệu từ hóa đơn. Nhờ vào việc tự động nhận dạng ký tự, giảm thiểu sự phụ thuộc vào công việc thủ công, giảm nguy cơ sai sót từ việc nhập liệu thủ công và tăng khả năng đáp ứng nhanh chóng các yêu cầu kinh doanh.

\section{Hạn chế} \label{hanche}
Tuy nhiên, mô hình còn một số hạn chế như sau:
\begin{enumerate}
    \item Dễ ảnh hưởng bởi nhiều yếu tố như chất lượng hình ảnh, font chữ, định dạng
    hóa đơn
    \item Mô-đun phát hiện văn bản vẫn chưa phát hiện được hầu hết văn bản. Điều này đẫn đến thiếu xót thông tin trong quá trình nhận diện và trích xuất thông tin.
    \item Mô-đun nhận diện văn bản, dễ bị ảnh hưởng bởi chất lượng hình ảnh, và nhiễu. Điều này làm cho quá trình nhận dạng trở lên sai xót ảnh hưởng đến thông tin lưu trữ.
    \item Mẫu hóa đơn chưa được đa dạng và phong phú, chỉ nhận dạng trên vào mẫu hóa đơn nhất định.
\end{enumerate}

\section{Định hướng phát triển}
Trong tương lai, Để để tài có thể phát triển thành một ứng dụng hoàn chỉnh, Em đề xuất giải pháp như sau:
\begin{enumerate}
    \item Khắc phục các hạn chế ở \ref{hanche}, tối ưu hóa độ chính xác, thời gian chạy và đa dạng mẫu hóa đơn ,và cải thiện khả năng nhận dạng trong các trường hợp phức tạp.
    \item Xây dựng ứng dụng di động cho phép người dùng chụp ảnh hóa đơn bằng điện thoại di động và tự động đưa ra thông tin liên quan. Điều này sẽ giúp tạo ra một trải nghiệm tiện lợi và linh hoạt cho người dùng.
    \item Tích hợp ứng dụng OCR vào hệ thống quản lý tài liệu và dữ liệu của doanh nghiệp. Điều này sẽ giúp tự động hóa quy trình làm việc, giảm thiểu thời gian và nguy cơ sai sót trong việc nhập liệu thủ công.
\end{enumerate}

Tổng kết lại, ``nghiên cứu về ứng dụng OCR trong nhận dạng hóa đơn'' là một bước tiến quan trọng trong việc áp dụng công nghệ vào thực tiễn kinh doanh. Tin rằng sự phát triển của OCR cùng với sự tương tác với các công nghệ khác sẽ mở ra những cơ hội mới và đóng góp tích cực cho sự phát triển của nền kinh tế và xã hội.




