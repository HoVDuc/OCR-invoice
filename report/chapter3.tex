\chapter{Phương pháp thực hiện}
Chương này sẽ trình bày các phương pháp và quy trình mà em sử dụng để thực hiện. Đây là phần quan trọng trong đề tài, nơi mô tả cách tiến hành thu thập dữ liệu, xây dựng mô hình, đánh giá hiệu suất và ứng dụng thực tế của công nghệ OCR trong việc nhận dạng và xử lý hóa đơn.

Dưới đây là các bước trình bày chi tiết:
\begin{enumerate}
    \item \textbf{Phầm mềm và công cụ hỗ trợ:} Trình bày các phầm mềm và công cụ cho quá trình chuẩn bị dữ liệu, xây dựng và huấn luyện mô hình.
    \item \textbf{Tạo môi trường và cài công cụ cần thiết}: Trong phần này, sẽ mô tả quá trình cài đặt môi trường cho đề tài và thiết lập một số môi trường cần thiết để chạy.
    \item \textbf{Thu thập dữ liệu:} Mô tả quy trình thu thập dữ liệu, bao gồm cách chọn các hình ảnh hóa đơn, cách đảm bảo tính đa dạng và độ phủ của dữ liệu, và việc chuẩn bị dữ liệu cho quá trình huấn luyện và kiểm tra mô hình.
    \item \textbf{Huấn luyện và tinh chỉnh:} Mô tả cách huấn luyện mô hình trên tập dữ liệu đã thu thập. Điều chỉnh tham số, tối ưu hóa và giải quyết vấn đề overfitting hoặc underfitting.

\end{enumerate}

Chương ``Phương pháp thực hiện'' là nền tảng quan trọng cho việc thực hiện nghiên cứu, giúp định hình cách tiến hành các bước quan trọng trong đề tài ``Nghiên cứu ứng dụng công nghệ OCR nhận dạng hóa đơn''.

\section{Phầm mềm và công cụ hỗ trợ}
\subsection{Google Colab}
Google Colab (viết tắt của Google Colaboratory) là một dịch vụ miễn phí của Google cho phép thực hiện và chia sẻ các tệp notebook Jupyter, cũng như code Python. Nó là một môi trường trực tuyến, cho phép người dùng viết và chạy code Python một cách trực tiếp trên trình duyệt web mà không cần cài đặt bất kỳ môi trường phát triển nào trên máy tính.

Colab cung cấp sử dụng miễn phí cho CPU, GPU và RAM để người dùng có thể thực hiện các nhiệm vụ tính toán phức tạp mà không cần phải mua hoặc cấu hình phần cứng riêng. Colab còn được tích hợp sẵn với nhiều thư viện phổ biến cho khoa học dữ liệu, học máy và xử lý ảnh, giúp dễ dàng tiến hành các tác vụ phức tạp. Có thể tạo notebook Jupyter, trong đó bạn có thể viết code Python từng cell và thực thi chúng một cách tương tác. Điều này rất hữu ích cho việc thử nghiệm, phân tích dữ liệu và xây dựng mô hình máy học.

Colab có thể chia sẻ notebook của mình với người khác thông qua liên kết. Người khác có thể xem và chỉnh sửa notebook hoặc thậm chí làm việc chung với nhau trên cùng một notebook. Có thể lưu notebook và dữ liệu của mình trực tiếp vào Google Drive để truy cập dễ dàng và chia sẻ với các thiết bị khác.

Google Colab thường được sử dụng trong việc học, nghiên cứu và phát triển các dự án liên quan đến khoa học dữ liệu, học máy và trí tuệ nhân tạo mà không cần đầu tư nhiều vào cấu hình phần cứng.

\subsection{PaddleOCR}
PaddleOCR là một dự án mã nguồn mở do PaddlePaddle phát triển, nhằm cung cấp một giải pháp toàn diện cho các nhiệm vụ liên quan đến xử lý ảnh và văn bản, bao gồm cả nhận dạng ký tự, nhận dạng văn bản và các tác vụ liên quan đến OCR. Dự án này được xây dựng trên cơ sở của các mô hình học sâu và sử dụng các thuật toán tiên tiến để giải quyết các thách thức trong việc xử lý ảnh và văn bản.

PaddleOCR hỗ trợ nhiều tác vụ liên quan đến OCR như nhận dạng ký tự, nhận dạng văn bản và phân loại chữ viết tay. Có thể được đào tạo và sử dụng cho nhiều ngôn ngữ khác nhau, giúp phát triển ứng dụng OCR toàn cầu. Người dùng có thể tùy chỉnh và đào tạo lại các mô hình của PaddleOCR cho phù hợp với nhu cầu cụ thể của dự án.

PaddleOCR có thể hoạt động trên nhiều nền tảng khác nhau, bao gồm máy tính cá nhân, máy chủ và các môi trường đám mây. Dự án cung cấp một loạt các mô hình học sâu đã được đào tạo trước để giúp giải quyết các vấn đề liên quan đến xử lý ảnh và văn bản.

Dự án được tối ưu hóa để đạt hiệu suất cao và đáp ứng yêu cầu xử lý ảnh và văn bản trong thời gian thực.

Hơn nữa PaddleOCR là một dự án mã nguồn mở và có cộng đồng hỗ trợ sẵn sàng chia sẻ kiến thức, giải đáp thắc mắc và cùng nhau phát triển.

Tóm lại, cung cấp một giải pháp mạnh mẽ và linh hoạt cho các ứng dụng liên quan đến xử lý ảnh và văn bản, đặc biệt là trong lĩnh vực OCR. Điều này giúp đơn giản hóa và tối ưu hóa quá trình xây dựng hệ thống nhận dạng văn bản và thông tin từ các hình ảnh hóa đơn và tài liệu khác.

\subsection{PPOCRLabel}
PPOCRLabel là một công cụ hỗ trợ trong lĩnh vực xử lý ảnh và trí tuệ nhân tạo, được sử dụng để thực hiện công việc nhận dạng và đánh dấu vùng chứa văn bản trên ảnh. Đây là một dự án mã nguồn mở của PaddlePaddle, một thư viện học máy phát triển bởi Baidu. PaddleOCR nhằm mục tiêu xây dựng các mô hình nhận dạng ký tự trên ảnh với hiệu suất cao.

PPOCRLabel được tạo ra để hỗ trợ quá trình chuẩn bị dữ liệu cho việc huấn luyện mô hình nhận dạng văn bản. Việc chuẩn bị dữ liệu là một bước quan trọng trong quá trình phát triển mô hình học máy, và công cụ như PPOCRLabel giúp đơn giản hóa và tăng cường hiệu suất của quá trình này. Dưới đây là một số tính năng chính của PPOCRLabel \cite{ppocrlabel}:
\begin{enumerate}
    \item \textbf{Labeling vùng chứa văn bản:} PPOCRLabel cho phép người dùng vẽ các hộp giới hạn xung quanh các vùng chứa văn bản trên ảnh để đánh dấu vị trí của văn bản cần nhận dạng.
    \item \textbf{Labeling trích xuất từ khóa:} Người dùng có thể gán nhãn thông tin từ khóa để cho bài toán trích xuất thông tin chính.
    \item \textbf{Gắn nhãn văn bản:} Người dùng có thể gắn nhãn văn bản được nhận dạng trong các hộp giới hạn để cho biết nội dung của văn bản đó.
    \item \textbf{Chú thích cho bảng:} PPOCRLabel cung cấp cho người dùng chức năng chú thích bảng nhằm mục đích bóc tách cấu trúc của bảng dưới dạng hình ảnh và chuyển sang định dạng Excel
    \item \textbf{Chỉnh sửa và xem trước:} PPOCRLabel cung cấp giao diện để chỉnh sửa và xem trước dữ liệu đã được đánh dấu trên ảnh, đảm bảo rằng dữ liệu được chuẩn bị chính xác trước khi sử dụng để huấn luyện mô hình.
    \item \textbf{Xuất dữ liệu:} Sau khi hoàn thành việc đánh dấu và chuẩn bị dữ liệu, PPOCRLabel cho phép bạn xuất dữ liệu trong các định dạng phổ biến để sử dụng trong quá trình huấn luyện mô hình.
    \item \textbf{Tích hợp với PaddleOCR:} PPOCRLabel có thể liên kết với dự án PaddleOCR để tiện lợi trong việc sử dụng dữ liệu đã được chuẩn bị để huấn luyện các mô hình nhận dạng văn bản.
\end{enumerate}

PPOCRLabel là một công cụ cực kỳ hữu ích trong quá trình chuẩn bị dữ liệu cho việc huấn luyện mô hình liên quan đến OCR, giúp tăng cường hiệu suất và chính xác của mô hình cuối cùng.

\section{Tạo môi trường và cài công cụ cần thiết}
\subsubsection*{Cài đặt Miniconda:}
Miniconda là một công cụ quản lý môi trường và gói thư viện Python nhẹ và tối giản. Dưới đây là hướng dẫn cách tạo môi trường Miniconda trên hệ điều hành Ubuntu:

\textbf{Bước 1:} Tải Miniconda
\begin{lstlisting}[language=bash]
    wget https://repo.anaconda.com/miniconda/Miniconda3-latest-Linux-x86_64.sh
\end{lstlisting}

\textbf{Bước 2:} Cài đặt Miniconda
\begin{lstlisting}[language=bash]
    sh Miniconda3-latest-Linux-x86_64.sh    
\end{lstlisting}
Chấp nhận tất cả các điểu khoản
\begin{lstlisting}[language=bash]
    source ~/.bashrc
\end{lstlisting}

\textbf{Bước 3:} Tạo môi trường
\begin{lstlisting}[language=bash]
    conda create -n paddle_env python=3.8
\end{lstlisting}
Nhấn y để cài đặt

\textbf{Bước 4:} Kích hoạt môi trường
\begin{lstlisting}[language=bash]
    conda activate paddle_env
\end{lstlisting}
\begin{figure}[h]
    \includegraphics[scale=0.5]{images/terminal-conda-activate.png}    
    \centering
    \caption{Môi trường đã được kích hoạt}
\end{figure}

\subsubsection*{Cài đặt PaddleOCR}
\textbf{Bước 1:} Cài đặt PaddlePaddle
\begin{lstlisting}[language=bash]
    python3 -m pip install paddlepaddle -i https://mirror.baidu.com/pypi/simple
\end{lstlisting}

\textbf{Bước 2:} Clone PaddleOCR và cài đặt
\begin{lstlisting}[language=bash]
    git clone https://github.com/PaddlePaddle/PaddleOCR.git
    pip install -r PaddleOCR/requirements.txt
\end{lstlisting}

\subsubsection*{Cài đặt PPOCRLabel}
\begin{lstlisting}[language=bash]
    pip3 install PPOCRLabel
    pip3 install trash-cli
\end{lstlisting}


\section{Thu thập và chuẩn bị dữ liệu huấn luyện}
\subsection{Lựa chọn hóa đơn}
Trong phần này, em xác định loại hóa đơn mà nghiên cứu sẽ tập trung nhằm mục tiêu ứng dụng công nghệ OCR trong việc nhận dạng và trích xuất thông tin. Loại hóa đơn được chọn là hóa đơn bán hàng ở cửa hàng tiện lợi và siêu thị, là một phần quan trọng của cuộc sống hàng ngày và có sự đại diện cho nhiều thông tin đa dạng cần được xử lý.

Hóa đơn bán hàng cửa hàng tiện lợi và siêu thị được lựa chọn vì nó thường chứa các thông tin cần thiết như địa chỉ, ngày mua, sản phẩm, số lượng, cũng như các khoản phí.\ldots Loại hóa đơn này thường đa dạng về cấu trúc và kiểu dáng, bao gồm vùng văn bản in và cả phần hình ảnh với các dữ liệu chú thích. Do đó, ứng dụng công nghệ OCR để tự động nhận dạng và trích xuất thông tin từ loại hóa đơn này đem lại giá trị thực tiễn và hứa hẹn trong việc tối ưu hóa quá trình xử lý hóa đơn và quản lý tài liệu.

Để thực hiện đề tài này em sử dụng mẫu hóa đơn của cửa hàng tiện lợi Okono và VinMart (Hình \ref{fig9-okono-vincom}) để thực hiện trích xuất thông tin.

\begin{figure}[h]
    \includegraphics[scale=0.20]{images/okono-vincom.png}
    \centering
    \caption{Hóa đơn Okono(trái) và VinCommerce(phải)}
    \label{fig9-okono-vincom}
\end{figure}

Qua việc tập trung vào loại hóa đơn bán hàng cửa hàng tiện lợi và siêu thị, mong muốn thực hiện một nghiên cứu chi tiết về cách ứng dụng công nghệ OCR vào việc nhận dạng và xử lý dữ liệu từ hóa đơn trong ngữ cảnh thực tế.

\subsection{Chuẩn bị dữ liệu}
Chuẩn bị dữ liệu đây là một bước quan trọng để đảm bảo rằng mô hình OCR hoạt động tốt trên các dữ liệu thực tế. Trong phần này, em sẽ trình bày quá trình chuẩn bị và xử lý dữ liệu hình ảnh của hóa đơn. Tùy với nhiệm vụ khác nhau trong OCR ta có cách chuẩn bị dữ liệu khác nhau. Để thực hiện gán nhãn dữ liệu OCR, em sử dụng công cụ PPOCRLabel. Dưới đây là các bước thực hiện:

\textbf{Bước 1}: Chuẩn bị tập dữ liệu cần gãn nhãn
Ở bước này em chuẩn bị hơn 75 ảnh hóa đơn của cửa hàng tiện lợi Okono và xác định số lớp cần gãn nhãn cho bài toán KIE.  

\textbf{Bước 2}: Chạy chương trình PPOCRLabel
\begin{lstlisting}[language=bash]
    PPOCRLabel --kie True # [KIE mode] for [detection + recognition + keyword extraction] labeling
\end{lstlisting}
\begin{figure}[h]
    \includegraphics[scale=0.25]{images/UI-ppocr.png}
    \centering
    \caption{Giao diện PPOCRLabel}
\end{figure}

\textbf{Bước 3}: Nhấp vào nút "Open Dir" để mở thư mục chứa các hình ảnh cần gán nhãn. Các hình ảnh sẽ hiển thị trong danh sách.
\begin{figure}[h]
    \includegraphics[scale=0.26]{images/UI-ppocr-image.png}
    \centering
    \caption{Danh sách ảnh PPOCRLabel}
\end{figure}

\textbf{Bước 4}: Chọn một hình ảnh, sử dụng các công cụ vẽ để khoanh vùng văn bản và gán nhãn tương ứng. Có thể gán nhiều vùng văn bản cho một hình ảnh.
\begin{itemize}
    \item W: để tạo một hộp cho phát hiện văn bản.
    \item Ctrl + E: Sửa nhãn của hộp.
    \item Ctrl + X: Thay đổi lớp của nhãn dùng cho gãn nhãn KIE.
\end{itemize}
Ở đây số lớp của bài toán KIE em xác định bao gồm các lớp sau:
\begin{itemize}
    \item OTHER: 
    \item SELLER: Tên cửa hàng
    \item ADDRESS: Địa chỉ 
    \item TIMESTAMP: Thời gian mua
    \item STAFF: Nhân viên bán hàng
    \item PRODUCT: Sản phẩm đã mua
    \item NUMBER: Số lượng sản phẩm
    \item PRICE: Giá của sản phẩm
    \item TOTAL\_COST: Tổng tiền thanh toán
\end{itemize}
\begin{figure}[h]
    \includegraphics[scale=0.26]{images/GUI-ppocr-labeled.png}
    \centering
    \caption{Gán nhãn ảnh với PPOCRLabel}
\end{figure}

\textbf{Bước 5}: Lưu các thay đổi bằng nút "Save" hoặc tự động lưu khi thoát chương trình.

\textbf{Bước 6}: Tiếp tục gán nhãn cho các hình ảnh còn lại.

\textbf{Bước 7}: Nhấn nút "Export Recognition Result" để cắt ảnh cho tác vụ training nhận dạng văn bản.
\begin{figure}[h]
    \includegraphics[scale=0.5]{images/data-invoice-text-recognition.png}
    \centering
    \caption{Dữ liệu hóa đơn được export từ PPOCRLabel}
\end{figure}

\textbf{Bước 8}: Kết thúc trương trình

Sau khi gãn nhãn xong ta thu được một file Label.txt, file có format như sau: 
\begin{lstlisting}[language=bash]
    image_files/IMG20230626141447.jpg	[{"transcription": "Okono", "points": [[440, 195], [807, 201], [805, 333], [438, 327]], "difficult": false, "key_cls": "SELLER"}...]
    image_files/IMG20230628105753.jpg	[{"transcription": "Okono", "points": [[429, 469], [700, 469], [700, 542], [429, 542]], "difficult": true, "key_cls": "SELLER"}...]
\end{lstlisting}

Các điểm trong các từ điển đó biểu thị tọa độ $(x, y)$ của bốn điểm của hộp văn bản, được sắp xếp theo chiều kim đồng hồ từ điểm ở góc trên bên trái. Trong transcription đại diện cho văn bản trong hộp, và key\_cls đại diện cho lớp của văn bản.

Dữ liệu cho nhận dạng văn bản khi export từ PPOCRLabel có format như sau:
\begin{lstlisting}[language=bash]
    #Tên ảnh                      #Thông tin chú thích ảnh
    IMG20230626141447_crop_0.jpg	Okono
    IMG20230626141447_crop_2.jpg	PHIẾU BÁN HÀNG/ INVOICE
    IMG20230626141447_crop_5.jpg	Mã số BBB
    IMG20230626141447_crop_6.jpg	Tên hàng
    IMG20230626141447_crop_7.jpg	SL
    ...
\end{lstlisting}
Để cho đúng dữ liệu cho bài toán này ta cần đưa nó về chuẩn LMDB:

\subsection{Tổng quan về dữ liệu đào tạo}
Dữ liệu cho nhiệm vụ phát hiện văn bản và KIE có cấu trúc như sau:
\begin{lstlisting}[language=bash]
|-train_data
    |- invoice
        |- train
            |- image_files
                |- image_001.jpg
                |- image_002.jpg
                | ...
            | train.txt
        |- valid
            |- image_files
                |- image_001.jpg
                |- image_002.jpg
                | ...
            | valid.txt  
\end{lstlisting}

Bao gồm 
Dữ liệu cho nhiệm vụ phát hiện văn bản và KIE có cấu trúc như sau:
\begin{lstlisting}[language=bash]
    |-train_data
        |- invoice-rec
            |- images
                |- image_001.jpg
                |- image_002.jpg
                | ...
            |- train.txt
            |- valid.txt
    \end{lstlisting}

\section{Huấn luyện mô hình}
Để huấn luyện mô hình OCR cho nhiệm vụ nhận diện hóa đơn đây là một quá trình phức tạp đòi hỏi sự kết hợp giữa nhiều nhiệm vụ khác nhau, mỗi nhiệm vụ là một quá trình huấn luyện riêng biệt. Điều này đảm bảo chất lượng của quá trình nhận dạng như Phát hiện vùng văn bản, Nhận dạng văn bản và Hiểu tài liệu(Document Understanding).

\subsection{Cấu hình phần cứng huấn luyện}
Với nhiệm vụ Phát hiện văn bản và KIE em sử dụng Colab pro để thực hiện huấn luyện mô hình:
\begin{itemize}
    \item Hệ điều hành: Linux
    \item Ngôn ngữ: Python 3.10
    \item CUDA: 11.7
    \item Framework: PaddlePaddle 1.4.2
    \item RAM: 13GB
    \item GPU: NVIDIA T4 - 16G
\end{itemize}

Còn đối với nhiệm vụ Nhận dạng văn bản em sử dụng server của vast.ai để thực hiện huấn luyện mô hình:
\begin{itemize}
    \item Hệ điều hành: Linux
    \item Ngôn ngữ: Python 3.8
    \item CUDA: 11.7
    \item Framework: Python 2.0.1
    \item CPU: AMD Ryzen Threadripper 1900X 8-Core Processor
    \item GPU: NVIDIA RTX 3060 - 12GB
    \item RAM: 64GB 
\end{itemize}

\subsection{Phát hiện văn bản}
Ở tác vụ này em lựa chọn thuật toán DBNet để phát hiện vùng văn bản. Sau khi quyết định được mô hình, download pre-trained model đã được train sẵn với dataset tiếng anh là ICDAR2015 dataset. Với backbone là ResNet50\_vd
\begin{lstlisting}[language=bash]
    # download the pre-trained model ResNet50_vd
    wget -P ./pretrain_models/ https://paddleocr.bj.bcebos.com/pretrained/ResNet50_vd_ssld_pretrained.pdparams
\end{lstlisting}

Bắt đầu training 
\begin{lstlisting}[language=bash]
python3 tools/train.py -c configs/det/det_r50_vd_db.yml  -o \
    Global.pretrained_model='./pretrain_models/ResNet50_vd_ssld_pretrained.pdparams' \
    Train.dataset.data_dir='./train_data/Okono_3/train/image/' \
    Train.dataset.label_file_list=['./train_data/Okono_3/train/train.txt'] \
    Train.loader.batch_size_per_card=12 \
    Eval.dataset.data_dir='./train_data/Okono_3/valid/image/' \
    Eval.dataset.label_file_list=['./train_data/Okono_3/valid/valid.txt'] \
    Global.save_model_dir='../drive/MyDrive/save/' \
    Global.checkpoints='/content/PaddleOCR/output/det_r50_vd' \
    Global.eval_batch_step=100 \
    Global.cal_metric_during_train=True \
\end{lstlisting}
\begin{lstlisting}[language=bash]
[2023/08/25 11:14:07] ppocr INFO: Architecture : 
[2023/08/25 11:14:07] ppocr INFO:     Backbone : 
[2023/08/25 11:14:07] ppocr INFO:         layers : 50
[2023/08/25 11:14:07] ppocr INFO:         name : ResNet_vd
[2023/08/25 11:14:07] ppocr INFO:     Head : 
[2023/08/25 11:14:07] ppocr INFO:         k : 50
[2023/08/25 11:14:07] ppocr INFO:         name : DBHead
[2023/08/25 11:14:07] ppocr INFO:     Neck : 
[2023/08/25 11:14:07] ppocr INFO:         name : DBFPN
[2023/08/25 11:14:07] ppocr INFO:         out_channels : 256
[2023/08/25 11:14:07] ppocr INFO:     Transform : None
[2023/08/25 11:14:07] ppocr INFO:     algorithm : DB
[2023/08/25 11:14:07] ppocr INFO:     model_type : det
...
eval model:: 100% 31/31 [00:03<00:00,  2.97it/s]
[2023/08/25 13:22:35] ppocr INFO: cur metric, precision: 0.2653061224489796, recall: 0.17067833698030635, hmean: 0.20772303595206393, fps: 14.512030517908638
[2023/08/25 13:22:35] ppocr INFO: best metric, hmean: 0.8958, is_float16: False, start_epoch: 156, precision: 0.9292, recall: 0.8648, fps: 13.95, best_epoch: 280
\end{lstlisting}

\subsection{Phát hiện văn bản}
Với tác vụ nhận dạng văn bản em sử dụng VietOCR. Dưới đây là các bước huấn luyện mô hình:
\begin{lstlisting}[language=Python]
    from vietocr.tool.config import Cfg
    from vietocr.model.trainer import Trainer
\end{lstlisting}

Load config VGG19 + Transformer và điều chỉnh tham số:
\begin{lstlisting}[language=Python]
    config = Cfg.load_config_from_name('vgg_transformer')
    dataset_params = {
        'name':'hw',
        'data_root':'./train_data/invoice-rec/',
        'train_annotation':'train.txt',
        'valid_annotation':'valid.txt'
    }

    params = {
            'print_every':200,
            'valid_every':15*200,
            'iters':20000,
            'checkpoint':'./checkpoint/transformerocr_checkpoint.pth',    
            'export':'./weights/transformerocr.pth',
            'metrics': 10000
            }

    config['trainer'].update(params)
    config['dataset'].update(dataset_params)
    config['device'] = 'cuda:0'
\end{lstlisting}

Bắt đầu huấn luyện mô hình:
\begin{lstlisting}[language=Python]
    trainer.train()
\end{lstlisting}

\begin{lstlisting}[language=Python]
iter: 000200 - train loss: 2.627 - lr: 1.51e-05 - load time: 0.72 - gpu time: 97.80
iter: 000400 - train loss: 2.367 - lr: 2.45e-05 - load time: 0.06 - gpu time: 91.02
iter: 000600 - train loss: 2.212 - lr: 3.95e-05 - load time: 0.07 - gpu time: 93.54
...
iter: 029600 - train loss: 0.526 - lr: 1.63e-07 - load time: 0.06 - gpu time: 84.60
iter: 029800 - train loss: 0.541 - lr: 4.14e-08 - load time: 0.06 - gpu time: 90.35
iter: 030000 - train loss: 0.550 - lr: 1.20e-09 - load time: 0.07 - gpu time: 90.14
iter: 030000 - valid loss: 0.551 - acc full seq: 0.8101 - acc per char: 0.9543
\end{lstlisting}

\subsection{Key information extraction}
Cuối cùng là mô hình KIE, ở đây em sử dụng LayoutXLM SER để huấn luyện. Dưới đây là các bước:

Download pretrained model và giải nén.
\begin{lstlisting}[language=bash]
    mkdir pretrained_model
    cd pretrained_model
    wget https://paddleocr.bj.bcebos.com/ppstructure/models/vi_layoutxlm/ser_vi_layoutxlm_xfund_pretrained.tar
    tar -xf ser_vi_layoutxlm_xfund_pretrained.tar
\end{lstlisting}

Bắt đầu huấn luyện mô hình:
\begin{lstlisting}[language=bash]
python tools/train.py -c ./configs/kie/vi_layoutxlm/ser_vi_layoutxlm_xfund_zh.yml -o \
    Architecture.Backbone=&num_classes \
    Train.dataset.data_dir=train_data/invoice/train/image_files \
    Eval.datasetdata_dir=train_data/invoice/valid/image_files \
    PostProcess.class_path=train_data/invoice/class_list.txt
\end{lstlisting}

\begin{lstlisting}[language=bash]
[2023/08/14 17:40:07] ppocr INFO: Architecture : 
[2023/08/14 17:40:07] ppocr INFO:     Backbone : 
[2023/08/14 17:40:07] ppocr INFO:         checkpoints : None
[2023/08/14 17:40:07] ppocr INFO:         mode : vi
[2023/08/14 17:40:07] ppocr INFO:         name : LayoutXLMForSer
[2023/08/14 17:40:07] ppocr INFO:         num_classes : 17
[2023/08/14 17:40:07] ppocr INFO:         pretrained : True
[2023/08/14 17:40:07] ppocr INFO:     Transform : None
[2023/08/14 17:40:07] ppocr INFO:     algorithm : LayoutXLM
[2023/08/14 17:40:07] ppocr INFO:     model_type : kie
...
[2023/08/14 19:05:50] ppocr INFO: cur metric, precision: 0.9857, recall: 0.9928, hmean: 0.9892, fps: 24
[2023/08/14 19:08:10] ppocr INFO: save best model is to ./output/ser_vi_layoutxlm_xfund_zh/best_accuracy
[2023/08/14 19:08:10] ppocr INFO: best metric, hmean: 0.9857, precision: 0.9928, recall: 0.9892, fps: 24, best_epoch: 57
\end{lstlisting}

\subsection{Kết quả mô hình}
\begin{table}[h]
    \centering
    \begin{tabular}{ | c | c | c | c | c | c | } 
        \hline
        \textbf{Task} & \textbf{Task} & \textbf{Backbone} & \textbf{Hmean} & \textbf{Precision} & \textbf{Recall} \\ 
        \hline
        Detection & DBNet & ResNet50vd & $89.58\%$ & $92.92\%$ & $86.48\%$ \\ 
        \hline
        KIE & LayoutXLM & LayoutXLM & $98.57\%$ & $99.28\%$ & $98.92\%$ \\ 
        \hline
        \hline
        \textbf{Task} & \textbf{Task} & \textbf{Backbone} & \textbf{Accuracy} & \multicolumn{2}{c|}{\textbf{Accuracy char}}   \\ 
        \hline
        Recognition & VietOCR & VGG19 & $81.01\%$ & \multicolumn{2}{c|}{$95.43\%$} \\ 
        \hline
    \end{tabular}
    \caption{Độ chính xác của cả 3 mô hình}
    \label{table:1}
\end{table}


