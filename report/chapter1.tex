\section{Giới Thiệu}
\subsection{Lý do chọn đề tài}
Cuộc sống hiện nay việc mua bán trao đổi hàng hóa được diễn ra thường xuyên giữa người mua và người bán. Ban đầu hóa đơn có giá trị làm bằng chứng chứng nhận cho việc chuyển nhượng hàng hóa giữa hai bên, có giá trị làm bằng chứng chứng nhận cho việc chuyển nhượng hàng hoá giữa hai bên. Mọi việc tranh chấp trong mua bán hàng hoá hai bên tự giải quyết. \\
\indent Trong quá trình phát triển xã hội, hoá đơn được phổ biến dần trong một cộng đồng khi được cộng đồng chấp nhận một cách tự nguyện. Các cộng đồng có thể là các Phường hội hoặc các định chế làng, xã. Những tranh chấp trong việc mua bán hàng hoá được các cộng đồng xử lý trên cơ sở dân sự. Khi nhà nước tham dự vào quản lý mua bán hàng hoá và xử lý những tranh chấp về hàng hoá dựa trên pháp luật dân sự và hình sự thì hoá đơn được nhà nước quy định để làm căn cứ pháp lý chứng minh cho việc chuyển nhượng hàng hoá giữa các bên và làm căn cứ để xác nhận quyền sở hữu hợp pháp của người có hàng hoá. Do đó hóa đơn là một loại tài liệu quan trọng trong các giao dịch. Nó được sử dụng để ghi lại các giao dịch mua bán hàng hóa và dịch vụ. Thông tin trên hóa đơn bao gồm tên của người bán, tên của người mua, ngày lập hóa đơn, số lượng hàng hóa hoặc dịch vụ, giá cả, tổng số tiền phải thanh toán.v.v\dots \\
\indent Hiện nay, hóa đơn thông thường được lập dưới dạng tài liệu giấy, có thể là hóa đơn giá trị gia tăng, hóa đơn bán hàng, tem, vé, thẻ, phiếu thu tiền bảo hiểm\dots Hóa đơn giấy có thể được phát hành theo các hình thức như hóa đơn đặt in, hóa đơn tự in, hóa đơn mua của cơ quan thuế. Điều này gây ra một số khó khăn trong việc quản lý hóa đơn, chẳng hạn như:
\begin{itemize}
    \item Quá nhiều hóa đơn: Các doanh nghiệp có thể phát sinh một số lượng lớn hóa đơn, từ các nhà cung cấp, khách hàng và các bên liên quan khác. Việc quản lý nhiều hóa đơn có thể là một thách thức, đặc biệt nếu chúng không được tổ chức và lưu trữ một cách hiệu quả.
    \item Sai sót và mất mát: Quản lý hóa đơn thủ công có thể gặp phải sai sót và mất mát hóa đơn, đặc biệt khi các hóa đơn được lưu trữ và xử lý bằng tay. Điều này có thể dẫn đến việc đòi tiền sai, không thu được tiền đúng lúc hoặc mất cơ hội thu hồi tiền nợ. 
    \item Tìm kiếm hóa đơn: Khi cần tìm một hóa đơn cụ thể, việc tìm kiếm nó có thể là một thách thức nếu nó không được tổ chức và lưu trữ một cách hiệu quả. Điều này có thể dẫn đến chậm trễ trong quá trình thanh toán hóa đơn hoặc thậm chí mất hóa đơn.
    \item Lưu trữ hóa đơn. Các hóa đơn phải được lưu trữ trong một thời gian nhất định theo quy định của pháp luật. Điều này có thể là một thách thức nếu không có một quy trình lưu trữ hóa đơn hiệu quả.
    \item Tuân thủ luật pháp: Việc tuân thủ các quy định và luật pháp về hóa đơn là rất quan trọng. Nếu không tuân thủ đúng, doanh nghiệp có thể phải đối mặt với các vấn đề pháp lý và hậu quả tài chính nghiêm trọng.
    \item Thay đổi trong quy định thuế: Thay đổi trong quy định thuế và các quy tắc về hóa đơn có thể làm cho việc quản lý hóa đơn trở nên phức tạp hơn, đòi hỏi doanh nghiệp phải cập nhật và điều chỉnh quy trình của mình thường xuyên.
\end{itemize}

\indent Để giải quyết các vấn đề này ta có thể cân nhắc sử dụng các phần mềm quản lý hóa đơn hiện đại. OCR có thể giải quyết vấn đề này bằng cách tự động hóa quá trình nhập liệu của hóa đơn một cách đơn giản và dễ dàng.

\subsection{Giải pháp}
OCR là một công nghệ có thể giải quyết các vấn đề trên bằng cách tự động trích xuất thông tin từ hóa đơn. Đây là một công nghệ cho phép máy tính nhận dạng và chuyển đổi văn bản từ hình ảnh chứa văn bản thành dạng văn bản có thể chỉnh sửa, tìm kiếm và lưu trữ. Áp dụng OCR trong việc quản lý hóa đơn có thể giúp giải quyết một số vấn đề như sau:
\begin{itemize}
    \item \textbf{Tiết kiệm thời gian}: OCR giúp doanh nghiệp tiết kiệm lượng lớn thời gian so với quá trình nhập dữ liệu thủ công. Với công cụ OCR, thông tin có thể dễ dàng được trích xuất sang các định dạng kỹ thuật số theo nhu cầu chỉ bằng việc chụp và tải ảnh lên. Không chỉ vậy, dữ liệu khi được trích xuất có thể dễ dàng được tìm kiếm, chỉnh sửa và thực hiện nhiều tác vụ khác, hỗ trợ quy trình xử lý tài liệu dễ dàng và thuận tiện hơn. Trên thực tế, nghiên cứu đã phát hiện ra rằng lượng thời gian dành cho công việc giấy tờ có thể giảm 75\% khi sử dụng OCR. Trung bình, thời gian để trích xuất một tài liệu sang dạng số chỉ từ 0.5 – 2 giây với công cụ OCR, một sự tối ưu đáng kể so với thời giang trung bình 1– 5 phút khi sử dụng phương pháp nhập liệu truyền thống. \cite{fptai}
    \item \textbf{Cải thiện độ chính xác}: Việc nhập liệu bằng tay không chỉ tốn nhiều thời gian, nguồn lực mà còn có mức độ rủi ro cao trong sai sót nhập. Nhất là với các loại tài liệu bao gồm nhiều trường thông tin bằng số, địa chỉ email, địa chỉ nhà,... việc nhập tay thủ công khó có thể chính xác 100\%. Những lỗi sai thông tin ngay từ bước đầu sẽ khiến kho dữ liệu doanh nghiệp không được “sạch” và chính xác. 
    \item \textbf{Hỗ trợ tuân thủ luật pháp}: Sử dụng OCR giúp đảm bảo tính chính xác và toàn vẹn của dữ liệu trên hóa đơn, từ đó đảm bảo tuân thủ các quy định về hóa đơn và thuế.
    \item \textbf{Quản lý hóa đơn điện tử}: Kết hợp OCR với hóa đơn điện tử giúp tự động tạo và lưu trữ các hóa đơn điện tử, giảm thiểu việc sử dụng giấy tờ truyền thống và tiết kiệm không gian lưu trữ.
\end{itemize}

Nhìn chung, OCR là một công nghệ có nhiều tiềm năng ứng dụng trong lĩnh vực kế toán và tài chính. OCR có thể giúp các doanh nghiệp tiết kiệm thời gian, tăng cường độ chính xác và cải thiện khả năng truy xuất thông tin hóa đơn. Tuy nhiên, để OCR hiệu quả, ta cần đảm bảo rằng hóa đơn được quét và lưu trữ ở định dạng tốt, đủ để đảm bảo hiệu suất nhận dạng của OCR cao nhất. 

\subsection{Mục Tiêu}
Dựa vào những vấn đề của hóa đơn và các giải pháp của OCR ở mục 1.1 và 1.2, mục tiêu của đề tài “\textbf{Nghiên cứu ứng dụng công nghệ OCR nhận dạng hóa đơn}” là tìm hiểu, đánh giá khả năng ứng dụng của công nghệ OCR hiện nay trong việc quản lý hóa đơn. Cụ thể đề tài tập trung vào các mục tiêu sau:
\begin{itemize}
    \item Tìm hiểu về công nghệ OCR: Nghiên cứu các nguyên lý hoạt động của OCR, các phương pháp và thuật toán phổ biến trong việc nhận dạng văn bản từ hình ảnh.
    \item Phân tích hiệu quả và lợi ích của ứng dụng OCR trong quản lý hóa đơn: So sánh các phương pháp truyền thống và ứng dụng OCR trong việc quản lý hóa đơn, đánh giá hiệu quả và lợi ích mà OCR mang lại, bao gồm tối ưu hóa thời gian, giảm thiểu sai sót, tiết kiệm chi phí và tăng cường khả năng xử lý lượng hóa đơn lớn.
    \item Đề xuất giải pháp và quy trình triển khai OCR: Dựa trên kết quả nghiên cứu, đề xuất các giải pháp và quy trình triển khai OCR trong việc quản lý hóa đơn, bao gồm lựa chọn phần mềm OCR phù hợp, quy trình xử lý hóa đơn, quản lý dữ liệu và bảo đảm tính an toàn thông tin.
    \item Đánh giá hiệu quả thực tế: Tiến hành thử nghiệm ứng dụng OCR trong môi trường thực tế của doanh nghiệp hoặc tổ chức để đánh giá hiệu quả, tính ổn định và khả năng mở rộng của giải pháp OCR.
\end{itemize}
Dựa trên kết quả đánh giá, đề xuất các cải tiến và phát triển tương lai của công nghệ OCR trong việc quản lý hóa đơn, nhằm nâng cao hiệu quả và khả năng ứng dụng của nó trong thực tế

\subsection{Phạm vi đề tài}
Phạm vi của đề tài "Nghiên cứu ứng dụng OCR nhận dạng hóa đơn" tập trung vào việc áp dụng công nghệ Nhận dạng ký tự quang học (OCR) để tự động nhận dạng và trích xuất thông tin từ hình ảnh của hóa đơn. Giải quyết các vấn đề liên quan đến việc tự động hóa quá trình xử lý hóa đơn, giảm thiểu công việc thủ công và tối ưu hóa hiệu suất làm việc trong việc quản lý tài liệu.

Hiện nay, có rất nhiều mẫu hóa đơn chúng đều không theo một quy chuẩn cụ thể nào cả mỗi cửa hàng, doanh nghiệp lại có một loại hóa đơn riêng nên trong đề tài này em sẽ sử dụng hai loại hóa đơn chính để thực hiện đề tài này, mẫu hóa đơn của cửa hàng tiện lợi Okono và VinCom để thực hiện trích xuất các thông tin như tên cửa hàng, địa chỉ, thời gian mua hàng, tên nhân viên, sản phẩm đã mua, tổng tiền hàng,\ldots để thực hiện đề tài này.

\subsection{Bố cục chương}
\textbf{Chương 1: Giới thiệu đề tài} \\Chương này giới thiệu sơ lược qua sự hình thành và phát triển của hóa đơn, nêu qua vấn đề của hóa đơn trong cuộc sống hiện tại và cách OCR quá thể ứng dụng vào để giải quyết các vấn đề đó. Trình bày mục tiêu, giải pháp và phạm vi nghiên cứu của đề tài 

\textbf{Chương 2: Cơ sở lý thuyết} \\Trình bày các khái niệm cơ bản về OCR, Học sâu, lịch sử phát triển của và ứng dụng của Học sâu vào OCR. Giới thiệu về các thuật toán OCR như phát hiện văn bản, nhận diện văn bản và nhận dạng cấu trúc tài liệu.

\textbf{Chương 3: Phương pháp đề xuất}
