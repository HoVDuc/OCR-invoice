% Định nghĩa định dạng mới cho phần section =====================================================
\titleformat{\section}[hang]
{\centering\normalfont\Large\bfseries}
{\MakeUppercase{Chương \thesection:}}
{0.5em}
{\MakeUppercase}
[]

\titleformat*{\subsection}{\selectfont\bfseries}
\titleformat*{\subsubsection}{\selectfont\bfseries}


% Setting header and footer ======================================================================
\pagestyle{fancy}
\lhead{\fontsize{13pt}{15.6}\selectfont Báo cáo Đồ án tốt nghiệp}
\rhead{\fontsize{13pt}{15.6}\selectfont Trường Đại học Mỏ - Địa chất}
\lfoot{\fontsize{13pt}{15.6}\selectfont Hồ Văn Đức}
\cfoot{}
\rfoot{\fontsize{13pt}{15.6}\selectfont \thepage}
\renewcommand{\headrulewidth}{0.4pt}
\renewcommand{\footrulewidth}{0.4pt}

% Setting table of content ========================================================================
\renewcommand{\cfttoctitlefont}{\hfill\Large\bfseries}
\renewcommand{\cftaftertoctitle}{\hfill}
\renewcommand{\contentsname}{\MakeUppercase{Mục lục}}


% Setting font ====================================================================================
\renewcommand{\baselinestretch}{1.5} 

% Định nghĩa định dạng font Time New Roma =========================================================
% \makeatletter
% % \patchcmd{<cmd>}{<search>}{<replace>}{<success>}{<failure>}
% \patchcmd{\@makechapterhead}{\huge}{\large}{}{}% for \chapter
% \patchcmd{\@makechapterhead}{\Huge}{\large}{}{}% for \chapter
% \patchcmd{\@makeschapterhead}{\Huge}{\large}{}{}% for \chapter*
% \makeatother

% Setting sub section level 4 =====================================================================
\setcounter{secnumdepth}{4}

\titleformat{\paragraph}
{\normalfont\fontsize{13}{15.6}\bfseries}{\theparagraph}{1em}{}
\titlespacing*{\paragraph}
{0pt}{3.25ex plus 1ex minus .2ex}{1.5ex plus .2ex}

%

\setlist[itemize]{itemsep=0.1\baselineskip}

% Setting hyperlinks ============================================================================

\usetikzlibrary{arrows.meta} % for arrow size
\contourlength{1.4pt}

\tikzset{>=latex} % for LaTeX arrow head
\colorlet{myred}{red!80!black}
\colorlet{myblue}{blue!80!black}
\colorlet{mygreen}{green!60!black}
\colorlet{myorange}{orange!70!red!60!black}
\colorlet{mydarkred}{red!30!black}
\colorlet{mydarkblue}{blue!40!black}
\colorlet{mydarkgreen}{green!30!black}
\tikzstyle{node}=[thick,circle,draw=myblue,minimum size=22,inner sep=0.5,outer sep=0.6]
\tikzstyle{node in}=[node,green!20!black,draw=mygreen!30!black,fill=mygreen!25]
\tikzstyle{node hidden}=[node,blue!20!black,draw=myblue!30!black,fill=myblue!20]
\tikzstyle{node convol}=[node,orange!20!black,draw=myorange!30!black,fill=myorange!20]
\tikzstyle{node out}=[node,red!20!black,draw=myred!30!black,fill=myred!20]
\tikzstyle{connect}=[thick,mydarkblue] %,line cap=round
\tikzstyle{connect arrow}=[-{Latex[length=4,width=3.5]},thick,mydarkblue,shorten <=0.5,shorten >=1]
\tikzset{ % node styles, numbered for easy mapping with \nstyle
  node 1/.style={node in},
  node 2/.style={node hidden},
  node 3/.style={node out},
}
\def\nstyle{int(\lay<\Nnodlen?min(2,\lay):3)} % map layer number onto 1, 2, or 3

% Chỉnh caption
\captionsetup[table]{font={stretch=1.45}}     %% change 1.2 as you like
\captionsetup[figure]{font={stretch=1.45}}